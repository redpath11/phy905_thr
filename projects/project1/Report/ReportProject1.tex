\documentclass[10pt,showpacs,preprintnumbers,footinbib,amsmath,amssymb,aps,prl,twocolumn,groupedaddress,superscriptaddress,showkeys]{revtex4-1}
\usepackage{graphicx}
\usepackage{dcolumn}
\usepackage{bm}
\usepackage[colorlinks=true,urlcolor=blue,citecolor=blue]{hyperref}
\usepackage{color}

\newcommand{\deriv}[3][]{% \deriv[<order>]{<func>}{<var>}
	\ensuremath{ \frac{d^{#1} {#2}}{d {#3}^{#1}} } }

\begin{document}
\title{Project 1}
\author{Thomas Redpath}
\affiliation{Department of Physics, Michigan State University, Outer Space}
\begin{abstract}

We solve Poisson's equation numerically to examine the effects of tailoring a specialized algorithm on execution speed.
Our best algorithm runs as $4n$ FLOPS with $n$ the dimensionality of the matrix. We also explore the onset of
round-off errors and loss of precision as dimensionality becomes lareg.

\end{abstract}
\maketitle

\section{Introduction}

Poisson's equation plays a central role in electromagnetism as it governs how the electric potential $\Phi$ arrises from a
localized charge distribution $\rho (\mathbf{r})$. Poisson's equation for spherically symmetric $\Phi$ and $\rho (\mathbf{r})$,
where the substitution $\Phi(r) = \phi(r)/r$ had been made:

\begin{equation*}
	\deriv[2]{\phi}{r}= -4\pi r\rho(r).
\end{equation*}
Re-writting this equation to let $f$ stand for the combination of the r.h.s. terms, letting $\phi \rightarrow u$ and
$r \rightarrow x$ and taking Dirichlet boundary conditions yields

\begin{equation*}
	-u''(x) = f(x), \hspace{0.5cm} x\in(0,1), \hspace{0.5cm} u(0) = u(1) = 0.
\end{equation*}
The solution is easily approximated by discretizing the interval $(0,1)$ and using a numerical definition of the second
derivative.

\section{Theory, algorithms and methods}

\subsection{Discretizing Poisson's Equation}

The numerical definition of a second derivative is derived from taking a Taylor expansion of the function $u(x)$.
On a discretized interval, it may be expressed as

\begin{equation*}
	\deriv[2]{u}{x} _i \approx \frac{u_{i+1} + u_{i-1} - 2u_i}{h^2} + O(h^2)
\end{equation*}
where $h$ is the grid spacing and the subscripts index the grid points. With this notation, Poisson's equation may be
re-written as

\begin{equation*}
	- \frac{u_{i+1} + u_{i-1} - 2u_i}{h^2} = f_i
\end{equation*}
With this formulation, it is apparent that we can re-cast the discretized Poisson equation as a system of linear equations
for each grid point.

\begin{equation*}
	\mathbf{A} \mathbf{u} = h^2 \mathbf{f}
\end{equation*}
where $\mathbf{A}$ is a matrix with dimensionality given by the number of grid points $n$

\begin{equation*}
    \mathbf{A} = \begin{bmatrix}
                           2& -1& 0 &\dots   & \dots &0 \\
                           -1 & 2 & -1 &0 &\dots &\dots \\
                           0&-1 &2 & -1 & 0 & \dots \\
                           \dots & \dots   & \ddots &\ddots &\ddots & \dots \\
                           0&\dots & \dots & -1 &2& -1 \\
                           0&\dots & \dots & 0  &-1 & 2 \\
                      \end{bmatrix}
\end{equation*}
and $\mathbf{f}$ represents a column vector whose elements are given by the source function
evaluated at each of the grid points. For simplicity of notation, we will absorb the $h^2$ into the
column vector elements ($h^2 \mathbf{f} = \mathbf{b}$) so that the system of equations may
be written $\mathbf{Au} = \mathbf{b}$. By applying numerical methods to solve Poisson's
equation on a discretized grid with $n$ points, we are able to convert a second order differential
equation into a system of $n$ equations.

\subsection*{The General Algorithm}

The matrix $\mathbf{A}$ given above is a tridiagonal matrix with identical elements along
the main and ``1-off'' diagonals. Our first algorithm solves the system of equations for a general
tridiagonal matrix using Gaussian elimination to reduce the matrix to an upper diagonal matrix,
effectively solving for the last variable $u_n$, then backwards subtituting the result to solve
for the remaining unknowns.

\[
\begin{bmatrix}
	d_1 & a_1 & 0 & \dots & \dots & 0 \\
	c_1 & d_2 & a_2 & 0 & \dots & 0 \\
	0 & c_2 & d_3 & a_3 & 0 & \dots \\
	\vdots & & \ddots & \ddots & \ddots & \vdots \\
	\vdots & & & c_{n-2} & \ddots & a_{n-1}\\
	0 & & & & c_{n-1} & d_n \\
\end{bmatrix}
~
\begin{bmatrix}
	u_1\\
	u_2\\
	u_3\\
	u_4\\
	\vdots\\
	u_n\\
\end{bmatrix}
~
=
~
\begin{bmatrix}
	f_1\\
	f_2\\
	f_3\\
	f_4\\
	\vdots\\
	f_n\\
\end{bmatrix}
\]

The Gaussian elimination method uses the first row of the matrix to
eliminate elements below the main diagonal:
\begin{align*}
	\mathrm{R2} \rightarrow \tilde{R2}&:
	\begin{bmatrix}
		0 & (d_2 - a_1 c_1/d_1) & a_2 & 0 \dots & 0\\
	\end{bmatrix}
%%	\mathrm{R3} \rightarrow \tilde{R3}&:
%%	\begin{bmatrix}
%%		0 & 0 & d_3 - (c_2 / \tilde{d_2}) & a_3 - (c_2 / \tilde{d_2}) 0 \dots & 0 \\
%%	\end{bmatrix}
\end{align*}
The altered second row is subsequently used to eliminate below-diagonal terms in the row
below it until the solution for the $n^{\mathrm{th}}$ unknown is found. The
result may then be backsubstituted to solve for the rest of the unknowns.
Following this procedure, the diagonal elements are modified in the foward
substituion according to the formula

\begin{equation*}
	\tilde{d_i} = d_i - \frac{a_{i-1} c_{i-1}}{\tilde{d}_{i-1}}
\end{equation*}
and the right hand side elements are modified according to

\begin{equation*}
	\tilde{f_i} = f_i - \tilde{f}_{i-1} \frac{c_{i-1}}{\tilde{d}_{i-1}}
\end{equation*}

Reducing the matrix in this way gives

\begin{equation*}
	u_n = \frac{\tilde{f}_n}{\tilde{d}_n}
\end{equation*}
for the last unknown. The form for the backwards substition to determine the
rest of the unknowns it

\begin{equation*}
	u _i = \frac{\tilde{f}_i - a_i u_{i+1}}{\tilde{d}_i}
\end{equation*}

These three calculations must be made $n-1$ times to solve the problem. Considering
that there are three floating point operations (1 multiplication, 1 division,
1 subtraction) per calculation, we note that the compuatational requirements
for implementing this algorithm scale with the number of grid points $n$ as $9(n-1)$.

\subsection*{The Specialized Algorithm}

The procedure outlined above can be optimized by taking advantage of the form that
the matrix takes for the specific case of solving Poisson's equation
using the three-point formula. Since the tridiagonal matrix has 2 for all main
diagonal elements and $-1$ for all 1-off diagonal elements, we can make two
simplifications to the algorithm. First, we can precalculate the $\tilde{d}_i$
elements by noting that they are given by $(i+1)/i$. Second, we can substitute -1
in for the $a_i$ and $c_i$ elements. With these simplifications, the forward
substitution calculation reduces to

\begin{equation*}
	\tilde{f}_i = f_i + \frac{ \tilde{f}_{i-1}}{\tilde{d}_{i-1}}
\end{equation*}
and the backward substitution reduces to

\begin{equation*}
	u_i = \frac{ \tilde{f}_i + u_{i+1}} {\tilde{d}_i}
\end{equation*}
This streamlined algorithm requires 4 floating point operations per grid point.

\subsection*{LU Decomposition}

Another way to implement the Gaussian elimination is the LU Decomposition method.
This method involves decomposing our matrix $\mathbf{A}$ into a product of two
matrices: one with 1s on the main diagonal and nonzero elements only below the
main diagonal and a second matrix with nonzero elements only above the main
diagonal. This decomposition allows us to write our matrix equation in the form

\begin{equation*}
	\mathbf{A u} = \mathbf{LU u} = \mathbf{f}
\end{equation*}
which can be solved in two steps

\begin{equation*}
	\mathbf{U u} = \mathbf{y}; \hspace{0.5cm} \mathbf{L y} = \mathbf{f}
\end{equation*}

By construction the matrix $\mathbf{L}$ has an inverse (since its determinant
is equal to 1) that we can use to obtain

\begin{equation*}
	\mathbf{L}^{-1} \mathbf{u} = \mathbf{y}
\end{equation*}
The algorithms necessary to LU Decompose $\mathbf{A}$ are discussed in detail
in \citep{Morten} and \citep{Golub1996}. It can be shown that this process
requires on the order of $2/3 n^3$ floating point operations.

\section{Methods}

\subsection{Execution time}

We applied each of the procedures discussed in the previous section to solve
Poisson's equation. Table~\ref{tab:speedresults} compares the execution time
as a function of number of grid points for each algorithm.

\begin{table}
\centering
	\begin{tabular}{ c | c c c }
	 & \multicolumn{3}{c}{Time}\\
	$n$ & General & Specialized & LU Decomp\\
\hline
	10   & & & \\
	100  & & & \\
	1000 & & & \\
\hline
	\end{tabular}
	\caption{Summary of execution times for the algorithms used.}
	\label{tab:speedresults}
\end{table}

\subsection{Comparison to exact result}

For this project, we assume the source term $f(x) = 100 e ^{-10x}$.
This allows the solution to the differential equation to obtained as
$u(x) = 1 - (1 - e^{-10})x - e^{-10x}$. Direct substituion back into
the DE confirms that this is indeed the correct solution.

\begin{equation*}
	- u''(x) = - (-100 e^{-10x}) = 100 e^{-10x}
\end{equation*}

We can then compare the results of our numerical solutions directly
to the analytic results. Figure~\ref{fig:compexact} plots the results
of the generalized algorithm compared to the exact solution.

\begin{figure}
\centering
	\includegraphics{figures/sols.pdf}
	\caption{A comparison of the results from the numerical solution
	(with various choices for the number of grid points) to the exact
	result. The blue curve plots the numerical result with 10 grid
	points, the green curve uses 100 grid points and the red curve
	uses 1000 grid points. The black curve plots the exact result.
	Good agreement with the exact result is obtained once 100 grid
	points are used.}
	\label{fig:compexact}
\end{figure}

\section{Results and discussions}



\section{Conclusions}

\begin{thebibliography}{99}
%\bibitem{miller2006} G.~A.~Miller, A.~K.~Opper, and E.~J.~Stephenson, Annu.~Rev.~Nucl.~Sci.~{\bf 56}, 253 (2006).
\bibitem{Morten} M. Hjorth-Jensen, Computational Physics Lecture Notes Fall 2015, August 2015.
\bibitem{Golub1996} G. Golub, C. Van Loan, \textit{Matrix Computations} (John Hopkins University Press, 1996)
\end{thebibliography}

\end{document}
