\documentclass[10pt,showpacs,preprintnumbers,footinbib,amsmath,amssymb,aps,prl,twocolumn,groupedaddress,superscriptaddress,showkeys]{revtex4-1}
\usepackage{graphicx}
\usepackage{dcolumn}
\usepackage{bm}
\usepackage[colorlinks=true,urlcolor=blue,citecolor=blue]{hyperref}
\usepackage{color}

\newcommand{\deriv}[3][]{% \deriv[<order>]{<func>}{<var>}
	\ensuremath{ \frac{d^{#1} {#2}}{d {#3}^{#1}} } }

\begin{document}
\title{Project 1}
\author{Thomas Redpath}
\affiliation{Department of Physics, Michigan State University, Outer Space}
\begin{abstract}

We solve Poisson's equation numerically to examine the effects of tailoring a specialized algorithm on execution speed.
Our best algorithm runs as $4n$ FLOPS with $n$ the dimensionality of the matrix. We also explore the onset of
round-off errors and loss of precision as dimensionality becomes lareg.

\end{abstract}
\maketitle

\section{Introduction}

Poisson's equation plays a central role in electromagnetism as it governs how the electric potential $\Phi$ arrises from a
localized charge distribution $\rho (\mathbf{r})$. Poisson's equation for spherically symmetric $\Phi$ and $\rho (\mathbf{r})$,
where the substitution $\Phi(r) = \phi(r)/r$ had been made:

\begin{equation*}
	\deriv[2]{\phi}{r}= -4\pi r\rho(r).
\end{equation*}
Re-writting this equation to let $f$ stand for the combination of the r.h.s. terms, letting $\phi \rightarrow u$ and
$r \rightarrow x$ and taking Dirichlet boundary conditions yields

\begin{equation*}
	-u''(x) = f(x), \hspace{0.5cm} x\in(0,1), \hspace{0.5cm} u(0) = u(1) = 0.
\end{equation*}
The solution is easily approximated by discretizing the interval $(0,1)$ and using a numerical definition of the second
derivative.

\section{Theory, algorithms and methods}

\subsection{Discretizing Poisson's Equation}

The numerical definition of a second derivative is derived from taking a Taylor expansion of the function $u(x)$.
On a discretized interval, it may be expressed as

\begin{equation*}
	\deriv[2]{u}{x} _i \approx \frac{u_{i+1} + u_{i-1} - 2u_i}{h^2} + O(h^2)
\end{equation*}
where $h$ is the grid spacing and the subscripts index the grid points. With this notation, Poisson's equation may be
re-written as

\begin{equation*}
	- \frac{u_{i+1} + u_{i-1} - 2u_i}{h^2} = f_i
\end{equation*}
With this formulation, it is apparent that we can re-cast the discretized Poisson equation as a system of linear equations
for each grid point.

\begin{equation*}
	\mathbf{A} \mathbf{u} = h^2 \mathbf{f}
\end{equation*}
where $\mathbf{A}$ is a matrix with dimensionality given by the number of grid points $n$

\begin{equation*}
    \mathbf{A} = \begin{bmatrix}
                           2& -1& 0 &\dots   & \dots &0 \\
                           -1 & 2 & -1 &0 &\dots &\dots \\
                           0&-1 &2 & -1 & 0 & \dots \\
                           \dots & \dots   & \ddots &\ddots &\ddots & \dots \\
                           0&\dots & \dots & -1 &2& -1 \\
                           0&\dots & \dots & 0  &-1 & 2 \\
                      \end{bmatrix}
\end{equation*}
and $\mathbf{f}$ represents a column vector whose elements are given by the source function
evaluated at each of the grid points. For simplicity of notation, we will absorb the $h^2$ into the
column vector elements ($h^2 \mathbf{f} = \mathbf{b}$) so that the system of equations may
be written $\mathbf{Au} = \mathbf{b}$. By applying numerical methods to solve Poisson's
equation on a discretized grid with $n$ points, we are able to convert a second order differential
equation into a system of $n$ equations.

\subsection*{A General Algorithm}

The matrix $\mathbf{A}$ given above is a special tridiagonal matrix with identical elements along
the main and ``1-off'' diagonals. Our first algorithm solves the system of equations for a general
tridiagonal matrix:

\[
\begin{bmatrix}
	a_1 & m_1 & 0 & \dots & \dots & 0 \\
	c_1 & a_2 & m_2 & 0 & \dots & 0 \\
	0 & c_2 & a_3 & m_3 & 0 & \dots \\
	\vdots & & \ddots & \ddots & \ddots & \vdots \\
	\vdots & & & c_{n-2} & \ddots & m_{n-1}\\
	0 & & & & c_{n-1} & a_n \\
\end{bmatrix}
~
\begin{bmatrix}
	u_1\\
	u_2\\
	u_3\\
	u_4\\
	\vdots\\
	u_n\\
\end{bmatrix}
~
=
~
\begin{bmatrix}
	b_1\\
	b_2\\
	b_3\\
	b_4\\
	\vdots\\
	b_n\\
\end{bmatrix}
\]

\subsection*{A Specialized Algorithm}

\subsection*{LU Decomposition}

\section{Results and discussions}



\section{Conclusions}

\begin{thebibliography}{99}
\bibitem{miller2006} G.~A.~Miller, A.~K.~Opper, and E.~J.~Stephenson, Annu.~Rev.~Nucl.~Sci.~{\bf 56}, 253 (2006).
\end{thebibliography}

\end{document}