\documentclass[10pt,showpacs,preprintnumbers,footinbib,amsmath,amssymb,aps,prl,twocolumn,groupedaddress,superscriptaddress,showkeys]{revtex4-1}
\usepackage{graphicx}
\usepackage{dcolumn}
\usepackage{bm}
\usepackage[colorlinks=true,urlcolor=blue,citecolor=blue]{hyperref}
\usepackage{color}
\usepackage{amsmath}
\usepackage{multirow}
\usepackage{natbib}

\newcommand{\costa}[1]{% \costa{<power>}
	\ensuremath{\cos ^{#1} {\theta}} }
\newcommand{\sinta}[1]{% \sinta{<power>}
	\ensuremath{\sin ^{#1} {\theta}} }
\newcommand{\pwrten}[1]{%\pwrten{<power>}
	\ensuremath{10^{#1}} }
\newcommand{\rhomax}{
	\ensuremath{ \rho _{\mathrm{max}}} }
\newcommand{\fourpisqr}{
	\ensuremath{ 4 \pi ^2} }
\newcommand{\deriv}[3][]{% \deriv[<order>]{<function>}{<variable>}
	\ensuremath{ \frac{d ^{#1} {#2}}{d {#3}^{#1}}}}

\begin{document}
\title{PHY 905 Project 4: Molecular Dynamics}
\author{Thomas Redpath}
\affiliation{Department of Physics, Michigan State University}
\begin{abstract}

\end{abstract}
\maketitle

\section{Introduction}

Molecular dynamics


\section{Theory}

\subsection*{Lennard-Jones Potential}


\begin{equation}
	U(r_{ij}) = 4\epsilon\left[\left(\frac{\sigma}{r_{ij}}\right)^{12} - \left(\frac{\sigma}{r_{ij}}\right)^6\right],
	\label{eq:lj}
\end{equation}

\begin{figure}
	\centering
	\includegraphics[width=0.5\textwidth]{figures/LJ.pdf}
	\caption{The Lennard-Jones potential (blue) and derived force (red).}
	\label{fig:lj}
\end{figure}

\subsection*{Units}

The small scale of the system we are simulating lends itself to working in a
specialized set of units. We used the following four relations to define our
system of units

\begin{align*}
	\text{1 unit of mass } &= 1 \text{ a.m.u} = 1.661\times 10^{-27}\mathrm{kg},\\
	\text{1 unit of length } &= 1.0 {\AA} = 1.0\times 10^{-10}\mathrm{m},\\
	\text{1 unit of energy } &= 1.651\times 10^{-21}\mathrm{J},\\
	\text{1 unit of temperature} &= 119.735\mathrm{K}.
\end{align*}
Converting to this system of units, Boltzmann's constant becomes
$k = (1.38 \times \pwrten{-23} )(119.7)(1.65 \times \pwrten{-21}) = 1$.

\section*{Algorithms and Methods}

The C++ code developed for this project utilizes the velocity
Verlet method to solve Newton's equations of motion. In this section we briefly
summarize the derivation of this method from \citet{Morten}. We
then provide a description of the codes which can be
found at \url{https://github.com/redpath11/phy905_thr} in the
\texttt{projects/project4/src} directory.

\subsection*{Verlet Method}

We numerically solve the equations of motion using the velocity
Verlet method. Since we've assumed a Lennard-Jones type
pairwise interaction between atoms, the form of the acceleration
is known from the gradient of the interaction potential. The Verlet
method derives from a summation of two Taylor expansions. Consider
the position in one dimension

\begin{align*}
	x(t+h) &= x(t) + h x'(t) + \frac{1}{2} h^2  x''(t) + O(h^3)\\
	x(t-h) &= x(t)  - h x'(t) + \frac{1}{2} h^2  x''(t) + O(h^3).
\end{align*}
Summing these two expansions and switching to the discretized
notation gives

\begin{equation*}
	x_{i+1} = 2 x_i - x_{i-1} + h^2 x''_i + O(h^4).
\end{equation*}
In general, this algorithm is not self-starting since it requires the
value of $x$ at two previous points.

Now, consider a Taylor expansion of the velocity and note the
relationshipe between velocity and acceleration $v' = a$.

\begin{align*}
	v_{i+t} &= v_i + h v_i ' + \frac{1}{2} v_i '' + O(h^3)\\
	a_{i+1} &= v'_i + h v''_i + O(h^2).
\end{align*}
We will assume that the velocity is a linear function of time
$hv''_i \approx v'_{i+1} - v'_i$. We can now write the final
algorithmic form for the position and velocities.

\begin{align}
%%\begin{split}
	x_{i+1} &= x_i + hv_i + \frac{h^2}{2} v'_i + O(h^3)\\
	v_{i+1} &= v_i + \frac{h}{2}(v'_{i+1} + v'_i) + O(h^3)
\label{eq:Verlet}
%%\end{split}
\end{align}

\subsection*{Code}

\subsubsection*{Overview}

We were supplied a code skeleton (from
\url{ https://github.com/andeplane/molecular-dynamics-fys3150})
 outlining the major classes needed to run a molecular dynamics
simulation. From this starting point, we filled in details to apply
periodic boundary conditions, initialize the system of argon
atoms in face-centered cubic (FCC) lattice, calculate the pairwise
forces from the Lennard-Jones potential and compute physical
properites of the system. Figure~\ref{fig:flowchart} gives a
general overview of the code structure by identifying key
elements and where they reside in the within the class structure.

\begin{figure*}
	\includegraphics[width=\textwidth]{figures/flowchart.pdf}
	\caption{The flowchart gives a crude overview of how the
	code operates. Blue boxes represent classes and
	indicate processes carried out by one or more memeber
	functions. Green boxes represent functions crucial to the 
	simulation. Black arrows indicate the hierarchy of function
	calls for one timestep. Orange arrows depict function
	calls that result in writing data to a file.}
	\label{fig:flowchart}
\end{figure*}
	

\subsubsection*{Initialization}

Initialization of our system takes place in two general steps: (1)
setting the size, initial temperature and Lennard-Jones potential
characteristics and (2) setting the initial positions and velocities
for the atoms. The first step is implemented in the \texttt{main}
function. The size of the system is determined by the number of
unit cells in the FCC lattice and the lattice constant (the size of
one unit cell). We studied a system of argon atoms for which
the lattice constant is 5.26 {\AA} \textbf{REFERENCE}. The
$\epsilon,\sigma$ parameters specify the L-J potential.

The second initialization step is carried out in the \texttt{System}
class. This class contains a function \texttt{System::createFCCLattice}
to position $4 \times N_x \times N_y \times N_z$ atoms in a lattice and
give them random velocities drawn from a Maxwell-Boltzmann
distribution. The \texttt{System::removeTotalVelocity} function
subtracts the net velocity from each atom so that the center of mass
of the system is stationary.
These two functions are called in \texttt{main}.

\subsubsection*{Simulation}

\subsubsection*{Output}


\section{Results and discussion}

%\begin{figure}
%\centering
%	\includegraphics[width=0.5\textwidth]{figures/dnit2.pdf}
%	\caption{The number of similarity transformations needed to
%	diagonalize the matrix is plotted against the matrix
%	dimensionality. A quadratic fit gives a $2 N^2 - 8 N$ dependence
%	of the number of similarity transformations on the dimensionality
%	$N$.}
%	\label{fig:dnit}
%\end{figure}


%\begin{figure}
%\centering
%	\includegraphics[width=0.5\textwidth]{figures/gsw.pdf}
%	\caption{Ground state wavefunctions for different oscillator
%	strengths and no interaction between the electrons.
%	The weakest potential ($\omega = 0.01$) is shown in black and the
%	strongest potential ($\omega = 5$) is plotted in green.
%	The wavefunctions have been scaled to the $\omega = 5$ wavefunction.}
%	\label{fig:gsw}
%\end{figure}

%\begin{figure}[hbtp]
%\includegraphics[scale=0.4]{test1.pdf}
%\caption{Exact and numerial solutions for $n=10$ mesh points.} 
%\label{fig:n10points}
%\end{figure}



%\begin{figure}[h!]
%\centering
%	\includegraphics[width=0.5\textwidth]{figures/CIwvfn.pdf}
%	\caption{Ground state wavefunctions vs. separation distance for two
%	Coulomb-interacting electrons in a harmonic oscillator potential.}
%	\label{fig:ci}
%\end{figure}

\section{Conclusions}

%\begin{thebibliography}{99}
%%\bibitem{miller2006} G.~A.~Miller, A.~K.~Opper, and E.~J.~Stephenson, Annu.~Rev.~Nucl.~Sci.~{\bf 56}, 253 (2006).
%\end{thebibliography}


\bibliographystyle{plainnat}
\bibliography{refs}

\end{document}